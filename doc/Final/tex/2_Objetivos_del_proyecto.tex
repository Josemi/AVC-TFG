\capitulo{2}{Objetivos del proyecto}

En este apartado voy a comentar los distintos objetivos funcionales, técnicos y personales del proyecto.

\section{Objetivos funcionales}
En este subapartado voy a contar cuales han sido los objetivos, tanto mios como de Apace, en cuanto a qué queríamos que hiciesen las distintas aplicaciones del proyecto.
\begin{itemize}
	\item
	Tener una aplicación Android que nos permita recoger los audios y las opciones adicionales de cada paciente, y poder subirlos para su posterior tratado y análisis, con finalidad de encontrar el mejor algoritmo de clasificación posible para la aplicación final.
	\item
	Realizar una aplicación Android para la interpretación de los audios de las personas que se posee un clasificador en el servidor. En la aplicación se tiene que poder modificar las opciones adicionales de los pacientes, almacenándose en el servidor para no tener que ponerlas en cada ejecución.
	\item
	Se quiere que ambas aplicaciones sean simples de usar y con un ejecución muy lineal.
	\item
	Se quiere que la aplicación final sea lo más accesible posible para que pueda ser usada hasta por compañeros con parálisis cerebral u otro tipo de discapacidad.
	\item
	Se quiere dar difusión mediática del proyecto, grabando vídeos, haciendo presentaciones, etc.
\end{itemize}

\section{Objetivos técnicos}
Los objetivos técnicos son los que he tenido en cuenta a lo largo del desarrollo en cuanto a las herramientas y el uso de estas.
\begin{itemize}
	\item
	Las dos aplicaciones, la de obtención de datos y la aplicación final, tienen una versión de API mínimo de 23 (a partir de Android 6.0), para poder abarcar un mayor número de dispositivos donde se puedan usar.
	\item 
	Para la clasificación se quiere usar Python, por el conocimiento que ya tengo y por el uso de librerías avanzadas en métodos de clasificación como puede ser \textit{Scikit-Learn}.
	\item
	Usar Flask [flask] como framework para el servidor.
	\item 
	Usar un repositorio Git para tener un seguimiento del desarrollo y un control de versiones usando GitHub.
	\item 
	Usar ZenHub para controlar las tareas y tiempos del proyecto.
	\item
	Realizar pruebas unitarias y de integración sobre la aplicación final.
	\item
	Realizar el desarrollo del proyecto siguiendo el modelo SCRUM, orienta a trabajo personal, sobre todo en la filosofía incremental de los productos.
\end{itemize}

\section{Objetivos personales}
En este apartado quiero comentar los objetivos propios que he tenido con este proyecto.
\begin{itemize}
	\item
	Poder ayudar a personas con parálisis cerebral.
	\item 
	Mejorar mi capacidad de comunicación y exposición, preparando diversas presentaciones.
	\item
	Usar el conocimiento obtenido a lo largo de la carrera.
	\item 
	Aprender a programar en Android, ya que es uno de los sistemas operativos más importantes a día de hoy.
	\item 
	Aprender a desarrollar y desplegar un servidor.
	\item 
	Aprender a usar nuevas herramientas de documentación, como \LaTeX.
	\item 
	Aprender a usar nuevas herramientas de diseño y programación.
\end{itemize}
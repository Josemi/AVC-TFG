\capitulo{1}{Introducción}

La \textbf{parálisis cerebral} es una discapacidad, normalmente originada en la fase de gestación del feto, que afecta al sistema motor de la persona, causando problemas en los movimientos y la postura. Esta discapacidad o conjunto de trastornos son \textbf{permanentes y no progresivos}. A menudo, estas personas padecen otro tipo de discapacidades. Estas suelen afectar al sistema nervioso causando limitaciones en la capacidad de usar los sentidos con los cuales percibir los estímulos exteriores y limitaciones en la capacidad intelectual. Por ello, a la parálisis cerebral se le puede llamar una pluridiscapacidad. \cite{aspacequees,aspacecyl,rosenbaum2007report}

Este conjunto de \textbf{discapacidades} producen una gran limitación en las tareas que las personas que lo sufren pueden realizar. Al ser parte de las discapacidades relacionadas con la postura y el sistema motor del cuerpo, las carencias en cuanto al movimiento son claras. Además, las discapacidades en el sistema nerviosos que pueden causar limitaciones sensoriales y cognitivas que pueden llevar a restricciones en la percepción del mundo exterior, en la conducta o en la \textbf{comunicación}.

La parálisis cerebral se origina principalmente en la gestación del feto y es causada por tres tipos de factores: \textbf{prenatales} causados por alteraciones en la coagulación o de la placenta, por traumatismos, por infecciones y por otras causas incluida la gestación múltiple, \textbf{perinatales} donde nos encontramos con diferentes causas entre las que destaca la prematuridad y por último, \textbf{postnatales}, a partir de más de 28 días después del nacimiento, donde encontramos causas como traumatismos craneales postnatales, paradas cardio-respiratorias, etc.\cite{crene,causas}

Como ya se ha comentado antes, la parálisis cerebral es una discapacidad que afecta a un gran número de personas, pero cada parálisis cerebral es diferente. Según ASPACE, 1 de cada 500 personas nacen en España con parálisis cerebral, lo que nos lleva a un total de 120.000 personas en todo el país.\cite{aspacedatos}

Hay distintos tipos, e incluso de clasificaciones, de parálisis cerebral. En el extremo de las personas más afectadas nos encontramos con las mayores carencias que he comentado, estas personas sufren de una gran limitación de movimiento y postura. Tienen que estar todo el día en una cama y si se quieren mover, montados en sillas especiales hechas a medida para no producirles ningún tipo de molestia ni de lesión. Uno de los mayores problemas que tienen las personas que padecen este grado de discapacidad es la limitación en la \textbf{comunicación}, ya que debido a sus carencias en la capacidad de moverse y de controlar estos movimientos solo pueden emitir \textbf{ruidos} o carcajadas, y en algunos casos ni siquiera se puede saber si tienen una intención comunicativa, es decir, emiten esos ruidos para intentar comunicarse, o simplemente son ruidos sin sentido.

Hoy en día se usan distintas aplicaciones y programas para que personas con parálisis cerebral se puedan comunicar \cite{jocomunico}, pero estas aplicaciones no pueden ser usadas por las personas gravemente afectadas debido a sus restricciones en el movimiento y control de su cuerpo. Esto es un gran problema ya que les excluye de usar estas herramientas para poder comunicarse. Como he comentado anteriormente, las personas gravemente afectadas solo emiten ruidos, los cuales pueden o no tener una intención comunicativa. Solo los familiares y los cuidadores más cercanos a la persona pueden llegar a intuir si ese ruido tiene alguna intención y, si la tiene, poder interpretarlo para saber si, por ejemplo, tiene hambre o le duele algo. El problema está en si esta persona con parálisis cerebral se tiene que quedar, por ejemplo, con un cuidador nuevo que no conoce ni puede interpretar estos sonidos.

Es por ello que comenzó el desarrollo de \textbf{AVC}, una aplicación Android accesible e intuitiva, que nos permite interpretar la emoción o la respuesta de una persona con parálisis cerebral gravemente afectada a partir de los sonidos que emite, gracias a métodos de clasificación de minería de datos.

\section{Estructura de la Memoria}
La presente memoria se compone de los siguiente apartados: 
\begin{itemize}
	\item 
		\textbf{Introducción:} descripción de la parálisis cerebral, planteamiento del problemas e introducción de la solución.
	\item
		\textbf{Objetivos del proyecto:} conjunto de objetivos funcionales, técnicos y personales que se han tenido a lo largo de este proyecto.
	\item
		\textbf{Conceptos teóricos:} explicación de nociones teóricas básicas para poder entender el proyecto.
	\item
		\textbf{Técnicas y herramientas:} conjunto de técnicas y herramientas esenciales en el desarrollo del proyecto.
	\item
		\textbf{Aspectos relevantes del desarrollo:} explicación temporal de los hechos más relevantes del proyecto.
	\item
		\textbf{Trabajos relacionados:} herramientas y proyectos semejantes.
	\item
		\textbf{Conclusiones:} conclusiones sobre el desarrollo del proyecto y posibles líneas futuras de trabajo.
\end{itemize}

\section{Estructura de los Anexos}
El documento con los anexos tiene los siguiente componentes:
\begin{itemize}
	\item
	\textbf{Plan del proyecto:} planificación temporal y viabilidad económica y legal del proyecto.
	\item
	\textbf{Requisitos:} diseño de los casos de usos, y de los requisitos funcionales y no funcionales.
	\item
	\textbf{Diseño:} explicación de las distintas fases de diseño seguidas en la aplicación.
	\item
	\textbf{Manual del programador:} manual por el cual un desarrollador podría poner en funcionamiento los elementos del proyectos o proseguir su implementación.
	\item
	\textbf{Manual de usuario:} conjunto de manual para la correcta instalación y ejecución de las aplicaciones desarrolladas en el proyecto.
\end{itemize}

\section{Materiales adjuntos}
Los elementos desarrollados y documentados durante el proyecto se pueden ver en el repositorio en \textit{GitHub}, \url{https://github.com/Josemi/AVC-TFG}.
Junto con esta memoria se entrega un CD con el siguiente contenido:
\begin{itemize}
	\item Memoria en formato pdf.
	\item Anexos en formato pdf.
	\item Aplicación para la recogida de datos, prototipo inicial.
	\item Aplicación para la recogida de datos, versión final.
	\item Aplicación AVC para poder interpretar los sonidos de las personas con las que se ha entrenado.
	\item Servidor en Flask para la administración de los datos y para el procesamiento y clasificación de los audios.
	\item Documentación \textit{javadoc}, de las tres aplicaciones \textit{Android}.
	\item Manuales de usuario para las tres aplicaciones.
	\item Presentaciones de las aplicaciones que se han usado para enseñar el proyecto en diversas situaciones.
	\item Vídeos e imágenes de las diversas presentaciones que se han realizado en el proyecto.
\end{itemize}	